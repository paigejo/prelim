\documentclass[xcolor=svgnames]{beamer}

\usetheme{umbc1} 
%\usecolortheme{default}

\usepackage{/Users/paigejo/mystyle_beamer}
\usepackage{multicol}
\usepackage{verbatim}
\usepackage{capt-of}
\usepackage{natbib}
\usepackage{url}
\usepackage{hyperref}
\hypersetup{colorlinks,linkcolor=,urlcolor=Blue}

% keep only the style and the slide number in the footer:
\setbeamertemplate{footline}[text line]{%
  \vbox{%
    \insertvrule{2.0pt}{umbc@decorations.fg!50!bg}%
    \begin{beamercolorbox}[wd=\paperwidth,ht=2.25ex,dp=1ex]{umbc@decorations}%
            \Tiny\hspace*{4.65in}\insertframenumber{} %/ \inserttotalframenumber\hspace*{2ex}\hspace{4mm}
    \end{beamercolorbox}}}


%\beamertemplatenavigationsymbolsempty
%\setbeamertemplate{footline}[frame number]

\title{~\\~\\Geostatistical inference under preferential sampling: introduction
\\
{\small By Peter Diggle, Raquel Menezes, and Ting-li Su}
}
\author[John Paige]{John Paige}
\date[\today]{\vspace{.3in} \small \\ \today}
\institute[University of Washington] % (optional, but mostly needed)
{
  \vspace{.1in} \\ Statistics Department\\
{\sc  University of Washington  }
}

\begin{document}
\setlength{\columnsep}{.5cm}


\frame{\titlepage}

%---------------------------------------------------------------------------------------------------------------------------------------

%\begin{frame}
 % \frametitle{Outline}
 % \tableofcontents
%\end{frame}


%----------------------------------------------------------------------------------------------------------------------------------------
%  Section 1
%----------------------------------------------------------------------------------------------------------------------------------------
\section{Introduction}

%\begin{frame}
%  \frametitle{Outline}
%  \tableofcontents[currentsection]
%\end{frame}

%\frame{\tableofcontents}

%\subsection{Introduction to Problem}

\begin{frame}
\frametitle{Introduction to the Problem}
\begin{itemize}
\item \emph{Geostatistics} involves modeling a process that is continuous in space measured at discrete locations
\item Often, data is assumed to be distributed randomly throughout the domain
\item What happens when the chance of sampling the process is tied to the value of the process itself?
\begin{itemize}
\item This is called \emph{preferential sampling}
\end{itemize}

\end{itemize}

\end{frame}
%----------------------------------------------------------------------------------------------------------------------------------------
\begin{frame}
\frametitle{Motivating Dataset}
\begin{columns}[l]

\column{2.5in}
\begin{itemize}
\item Lead concentration ($\mu g/g$ dry weight) in Galicia, Spain
\item Samples in 1997 are concentrated to the north, but samples in 2000 are on lattice
\item How to tell if data is sampled preferentially (and at what level)?
\item If so, how does preferential sampling effect `naive' inference?
\item How can we effectively take preferential sampling into account?
\end{itemize}

\column{2.5in}
\begin{figure}
\centering
\image{width=2.5in}{sampling_locations.pdf}
\end{figure}

\end{columns}

\end{frame}

%----------------------------------------------------------------------------------------------------------------------------------------
\begin{frame}
\frametitle{Preferential Sampling}
$S = \{S(x) : x \in \mathbb{R}^2\}$: spatially continuous stochastic process \\
$X = (x_1, ..., x_n)$: set of sample locations
~\\~\\
\emph{Preferential} sampling refers to when $\pi(S, X) \neq \pi(X) \pi(S)$ ($S$ is not independent of $X$).
~\\~\\
Note that non-preferential sampling does not necessitate uniform sampling.  Sample locations could still be clustered.
\end{frame}
%----------------------------------------------------------------------------------------------------------------------------------------
\begin{frame}
\frametitle{Scientific Context}

\begin{itemize}
\item \emph{Covariograms} defined the spatial structure of the covariance in $S$.  The traditional estimator is:
$$ \wh{C}(\mbf{h}) = \frac{1}{| N(\mbf{h})|} \sum_{N(\mbf{h})} (S(x_i) - \bbar{S})(S(x_j) - \bbar{S}) $$
where $x_i - x_j \approx \mbf{h}$ (under isotropic model, $|x_i - x_j| \approx h$)
\item \citealt{isaaks1988} and \citealt{srivastava1989} propose an alternative non-ergodic estimator:
$$ \wh{C}_{ne}(\mbf{h}) = \frac{1}{| N(\mbf{h})|} \sum_{N(\mbf{h})} (S(x_i) - \bbar{S}(\mbf{h}_i))(S(x_j) - \bbar{S}(\mbf{h}_j)) $$
where $\bbar{S}(\mbf{h}_i)$ is the sample mean of $S$ at all the points $x_i \in \mbf{h}_i$
\begin{itemize}
\item They claim this works better under preferential sampling
\end{itemize}
\item \citealt{curriero2002} shows this is ``equivalent'' yet ``worse'' than the traditional estimator under isotropy
\end{itemize}

\end{frame}
%----------------------------------------------------------------------------------------------------------------------------------------
\begin{frame}
\frametitle{Scientific Context}

\begin{itemize}
\item \citealt{schlather2004} proposes tests for prefential sampling assuming stationarity
\begin{itemize}
\item Assumption of stationarity may affect legitimacy of results
\end{itemize}
\end{itemize}

\end{frame}
%----------------------------------------------------------------------------------------------------------------------------------------
\begin{frame}
\frametitle{Model for Preferential Sampling}

Three assumptions for model:
\begin{enumerate}
\item $S$ is a stationary, mean zero Gaussian process
\item Conditional on $S$, $X$ is an inhomogeneous Poisson process wtih intensity
$$\lambda(x) = \exp{\alpha + \beta S(x)}$$
\item $Y_i \vert S, X \iid \Norm{\mu + S(x_i), \tau^2}$
\end{enumerate}
~\\
1 + 2 $\Rightarrow$ $X$ is a log Gaussian Cox process

\end{frame}
%----------------------------------------------------------------------------------------------------------------------------------------
\begin{frame}
\frametitle{Log-Gaussian Cox Processes}

\begin{columns}[l]

\column{.05in}
\column{2.45in}
A \emph{Cox process} is a stochastic point process satisfying:
\begin{itemize}
\item $\Lambda(x)$ is a random rate process
\item Conditioned on $\Lambda(x) = \lambda(x)$, a Cox process is an inhomogeneous Poisson process with rate $\lambda(x)$
\end{itemize}
~\\
A \emph{log Gaussian} Cox process also satisfies $\Lambda(x) = \exp{Z(x)}$, where $Z(x)$ is a Gaussian random field.

\column{2.45in}

\begin{figure}
\centering
\image{width=2.45in}{cox_proc_fig.pdf}
\caption{From \citealt{diggle2010}.  An example of a log Gaussian Cox process on unit square where $\beta=2$, $\alpha=1$, and $S$ has Mat\'{e}rn covariance.}
\end{figure}

\column{.05in}

\end{columns}

\end{frame}
%----------------------------------------------------------------------------------------------------------------------------------------
\begin{frame}
\frametitle{Spatial Covariance Model}

The Mat\'{e}rn correlation model, as a function of distance, $u$, is:
$$ \rho(u; \phi, \kappa) = \frac{1}{2^{\kappa - 1} \Gamma(\kappa)} (u/\phi)^\kappa K_\kappa(u/\phi) $$
$\kappa$: shape parameter \\
$\phi$: scale parameter \\
$K_\kappa$: modified Bessel function of the second kind, of order $\kappa$
~\\~\\
\begin{itemize}
\item $\rho$ is called the \emph{correlogram} when viewed as a function of distance
\item The covariance as a function of distance is the \emph{covariogram}
\item The variance as a function of distance is the \emph{variogram}
\end{itemize}
\end{frame}
%----------------------------------------------------------------------------------------------------------------------------------------
\begin{frame}
\frametitle{Testing Affect of Sample Designs}
\begin{figure}
\centering
\image{width=4.5in}{sim_design.pdf}
\caption{From \citealt{diggle2010}}
\end{figure}
Variogram estimation tested under 500 simulations from three sampling designs:
\begin{enumerate}
\item[a] Uniform
\item[b] Preferential ($\beta=2$)
\item[c] Clustered
\end{enumerate}

\end{frame}
%----------------------------------------------------------------------------------------------------------------------------------------
\begin{frame}
\frametitle{Fitting the Model}
Make gridded approximation of $S = \{S_0, S_1\}$
\begin{itemize}
\item $S_0$ are data
\item $S_1$ are the values at other grid points
\end{itemize}
\bal
L(\vt) &= \int \pi(Y \vert X, S) \pi(X \vert S) \pi(S) \ dS \\
&= \hdots \\
%&= \int \pi(X \vert S) \pi(Y \vert X, S) \frac{\pi(S \vert Y)}{\pi(S \vert Y)} \pi(S) \ dS \\
%&= \int \pi(X \vert S) \pi(Y \vert S_0) \frac{\pi(S \vert Y)}{\pi(S_0 \vert Y) \pi(S_1 \vert S_0, Y)} \pi(S) \ dS \\
%&= \int \pi(X \vert S) \pi(Y \vert S_0) \frac{\pi(S \vert Y)}{\pi(S_0 \vert Y) \pi(S_1 \vert S_0)} \pi(S_0, S_1) \ dS \\
%&= \int \pi(X \vert S) \pi(Y \vert S_0) \frac{\pi(S \vert Y)}{\pi(S_0 \vert Y)} \pi(S_0) \ dS \\
&= E_{S \vert Y} \brack{\pi(X \vert S) \frac{\pi(Y \vert S_0)}{\pi(S_0 \vert Y)} \pi(S_0)} \\
&\approx m^{-1} \sum_{j=1}^m \pi(X \vert S_j) \frac{\pi(Y \vert S_{0j})}{\pi(S_{0j} \vert Y)} \pi(S_{0j})
\eal
where $S_j$ is the $j$th conditional simulation of $S$ conditioned on $Y$.
\end{frame}
%----------------------------------------------------------------------------------------------------------------------------------------
\begin{frame}
\frametitle{Goodness of Fit}
\emph{Reduced second moment measure} (or $K$-function) for defined model is given by:
$$ K(s) = \pi s^2 + 2 \pi \int_0^s (\exp{\beta^2 \sigma^2 \rho(u; \kappa, \phi)} - 1) u \ du $$
\begin{itemize}
\item $s$ represents the maximum distance apart points can be
\item $\rho(u ; \phi) \defn \corr{S(x), S(x') \vert \phi, \kappa, |x - x'| = u}$
\item $\phi$: Mat\'{e}rn scale parameter
\item $\kappa$: Mat\'{e}rn smoothness parameter
\end{itemize}
\end{frame}
%----------------------------------------------------------------------------------------------------------------------------------------
\begin{frame}
\frametitle{Goodness of Fit}
Define test statistic:
$$ T = \int_0^{0.25} \frac{(\wh{K}(s) - K(s))^2}{\nu(s)} \ ds $$
\begin{itemize}
\item $\wh{K}(s)$: empirical $K$-function
\item $\nu(s) \defn \var{\wh{K}}$
\end{itemize}
\end{frame}
%----------------------------------------------------------------------------------------------------------------------------------------
\begin{frame}
\frametitle{Conclusions}
\begin{itemize}
\item Taking into account preferential sampling is important!
\begin{itemize}
\item In the simulations (albiet with high $\beta$), variograms estimated naively were estimated poorly
\end{itemize}
\item Uniform sampling performed best, then clustered, then preferential
\item Proposed class of models is flexible and values for $\beta$ can be tested directly with likelihood ratio test
\end{itemize}
\end{frame}
%----------------------------------------------------------------------------------------------------------------------------------------

%----------------------------------------------------------------------------------------------------------------------------------------
%  Section 6
%----------------------------------------------------------------------------------------------------------------------------------------
\section{References}
\begin{frame}
\tiny
\frametitle{References}
\bibliographystyle{te}
\bibliography{../prelim}
\end{frame}
%---------------------------------------------------------------------------------------------------------------------------------------



\end{document}








